\documentclass[10pt,letter]{article}
\usepackage{qjh-article}
\usepackage{geostreams}
\usepackage{epsfig}
\usepackage{multicol}

\newcommand{\RPSnxD}{\RPSnx{n}{x}}
%\newcommand{\RPSnx}[2]{\llcorner\pt{#1}^{\;\;\pt{#2}}\urcorner}
%\newcommand{\RPSnd}[2]{\llcorner\pt{#1}^{+\pt{#2}}\urcorner}
\newcommand{\RPSnx}[2]{\lfloor_{\pt{#1}}^{\;\;\pt{#2}}\rceil}
\newcommand{\RPSnd}[2]{\lfloor_{\pt{#1}}^{+\pt{#2}}\rceil}

\makeatletter
\def\roI{\pst@object{roI}}
\def\roI@i(#1){%
 \@ifnextchar({\roI@ii(#1)}{\roI@ii(0,0)(#1)}%)
 }
\def\roI@ii(#1)(#2)#3#4{\psframe@i(#1)(#2)%
 \uput{5pt}[ur](#1){#3}\uput{5pt}[dl](#2){#4}}
%\makeatspecial


\author{Quinn Hart $<$qjhart@ucdavis.edu$>$}
\title{Spatial Selection in GeoStreams}

\abstract{\emph{ Streaming database processing has a great deal of
    potential impact for remotely-sensed imagery.  Besides it
    typically large bandwidth, imagery has a number of attributes that
    are different from generic streaming data.  First, many queries
    want to select small spatial regions from a larger region.
    Secondly, the data is highly organized as it streams.  Finally,
    the data often arrives in a very chunk-like manner, where regular
    portions arrive simultaneously.  Often these chunks of data arrive
    with attributes associated to the chunk, applying in some way to
    the individual parts of the image.  In this paper we describe how
    we take advantage of both of these aspects in our query processor.
    We describe a methodology handle many spatial extent queries over
    a single stream, which minimizes the size of the processing system
    and in-flight data.}}

\begin{document}

\maketitle

\begin{multicols}{2}

\paragraph{Point Sets}

In this paper, we very often refer to pointsets that are rectangular
in space.  We describe two syntax shortcuts in our definitions of
pointsets.  These describe pointsets based on the minimum and maximum
points in the sets (the corners), or the minimum point plus a delta.
Assume the pointsets $\ps{P},\ps{Q} \subset \Z[n]$.  We define the following:

\begin{align*}
\ps{P} = \RPSnx{n}{x} &= {i \in \Z[2] \qquad \pt{n} \le \pt{i} \le \pt{x}}  \\
\ps{Q} = \RPSnd{n}{d} &= {i \in \Z[2] \qquad \pt{n} \le \pt{i} \le \pt{n}+\pt{d}} \\
& \text{where } \pt{n},\pt{x},\pt{d} \in \Z[n] \\
\pt{n} &< \pt{i} \equiv n_1 < i_1, n_2 < i_2, \ldots, n_n < i_n
\end{align*}

Note that the points $\pt{n}$ and $\pt{x}$ correspond to
$\bigwedge{\ps{P}}$ and $\bigvee{\ps{P}}$ respectively.

\paragraph{Geostream semantics}

We assume the following about our input geostream.  A geostream can be
thought of as a relation with an attribute that is an image.  The
image has the following restrictions; it is a rectangular point set,
in these examples in \Zp[2], though the image can have higher
dimensions, and the image arrives in row-scan order.  We also assume
that all other attributes of a geostream arrive before the image, and
that the rectangular pointset of the image can be determined at the
start of the image stream.  The image arrives in discrete pieces,
which we call ``chunks''.  In this example, we assume that a chunk
contains an integral number of complete rows of imagery.  These
assumptions are typical of mainy image formats, in particular they
correspond to the GOES GVAR format, described below, which is our
target application.

\paragraph{Joining Two Regions}

There are a number of ways that the user would like to intersect two
regions together. Typically, we are describing the methods of using a
region in a query to extract a region from an image.  Three methods
are shown below.  Each figure shows one region
$Q=\RPSnx{(5,5)}{(8,8)}$, intersecting three query point sets,
$q=\RPSnx{(1,1)}{(9,9)}, r=\RPSnx{(2,2)}{(7,7)},
s=\RPSnx{(3,3)}{(9,4)}$.  Figure~\ref{fig:restriction}a shows a domain
restriction query.  Here, all the points that are in both the query
region and the image region.  One interesting aspect is that the even
images with no overlap return results.  This is a feature, in fact,
the user may be very interested in images that do not have any
overlap.  Figure~\ref{fig:subset} only responds with images that
completely overlap, the region of interest, by including a subset test
on the pointsets.


\psset{unit=0.25}
\begin{figure*}[htbp]
  \centering
  \subfigure[input]{
    \begin{FramePic}[10,10]
   \roi[style=query](5,5)(8,8){$Q$}
   \roi[style=frame](1,1)(9,9){$q$}
   \roi[style=frame](2,2)(7,7){$r$}
   \roi[style=frame](3,3)(9,4){$s$}
 \end{FramePic}
}
\subfigure[$q|_Q$]{
  \begin{FramePic}[10,10]
   \roi[style=overlap](5,5)(8,8){$q|_Q$}
 \end{FramePic}
}
\subfigure[$r|_Q$]{
 \begin{FramePic}[10,10]
   \roi[style=overlap](5,5)(7,7){$r|_Q$}
 \end{FramePic}  
}
\subfigure[$t|_Q$]{
   \begin{FramePic}[10,10]
   \roi[style=overlap](5,5)(5,5){$t|_Q$}
 \end{FramePic}
}
  \caption{Select g.id,$g.i|_Q$ from $g$}
  \label{fig:restriction}
\end{figure*} 

\begin{figure*}[htbp]
  \centering
  \subfigure[input]{
    \begin{FramePic}[10,10]
   \roi[style=query](5,5)(8,8){$Q$}
   \roi[style=frame](1,1)(9,9){$q$}
   \roi[style=frame](2,2)(7,7){$r$}
   \roi[style=frame](3,3)(9,4){$s$}
 \end{FramePic}
}
\subfigure[$q|_Q$]{
  \begin{FramePic}[10,10]
   \roi[style=overlap](5,5)(8,8){$q|_Q$}
 \end{FramePic}
}
  \caption{Select g.id,$g.i|_Q$ from $g$ where $g.i \subset Q$}
  \label{fig:subset}
\end{figure*} 


Figure~\ref{fig:extension} shows how a user would use image extension to
verify each returned image has the same pointset.  In this example,
$\Bbb{O}$ is taken as an image that covers all of $\Z{2}$, with a
value of 0 for all points.  Unlike the domain restrictions, the
extension of a pointset onto $\Bbb{O}|_Q$, will always return an
image with pointset $Q$.  This is an important modification because
only images that share a common pointset can be combined to form a new
dimension.  For example, \emph{Select g.id,$g.i|_Q|^{\Bbb{O}|_Q}$ from
  $g$} would return a set of images with the same pointset, whereas
\emph{Select g.id,$compose(g.i|_Q|^{\Bbb{O}|_Q})$ from $g$}, would
return a new image with an additional dimension.  Where,
$compose(x_1,x_2,\ldots,x_n)$ joins $n$ $k$ dimensional images with a
common point set, into a $k+1$ dimensional image.

\emph{ That's not very good notation, and there is an interesting
  point here.  In most SQL statements we have the notion of
  aggregation functions, where as in streaming databases, what we
  really want to stream are accumulation functions.  We should make a
  note that this is something that we need to study in more detail.}

\begin{figure*}[htbp]
  \centering
  \subfigure[input]{
    \begin{FramePic}[10,10]
   \roi[style=query](5,5)(8,8){$Q$}
   \roi[style=frame](1,1)(9,9){$q$}
   \roi[style=frame](2,2)(7,7){$r$}
   \roi[style=frame](3,3)(9,4){$s$}
 \end{FramePic}
}
\subfigure[$q|_Q$]{
  \begin{FramePic}[10,10]
   \roi[style=overlap](5,5)(8,8){$q|_Q$}
 \end{FramePic}
}
\subfigure[$r|_Q$]{
 \begin{FramePic}[10,10]
   \roi[fillstyle=solid,fillcolor=lightgray](5,5)(8,8){$Q$}
   \roi[style=overlap](5,5)(7,7){$r|_Q$}
 \end{FramePic}  
}
\subfigure[$t|_Q$]{
   \begin{FramePic}[10,10]
   \roi[fillstyle=solid,fillcolor=lightgray](5,5)(8,8){$Q$}
   \roi[style=overlap](5,5)(5,5){$t|_Q$}
 \end{FramePic}
}
  \caption{Select g.id,$i|_Q|^{\Bbb{0}|_Q}$ from $g$}
  \label{fig:extension}
\end{figure*} 

\paragraph{Multiple Spatial Queries}

Having described some simple overlaps above, we investigate a
structure to handle multiple queries.  Figure~\ref{fig:multi}a shows a
region with 3 outstanding querie pointsets.  Our goal is to rechunk
the input data so that for each query they satisify the same
constraints, that is they are in row scan order.  Additionally, we do
not want to replicate any parts of input chunk.  The strategy we use
is to rechunk our input data into smaller sizes, and assign to each
new chunk all the queries which it satisfies.  Since our input stream
is in row scan order it is simple to assign a proper order to new
chunks.

\psset{unit=3.0}
\begin{figure*}[htbp]
  \centering
  \subfigure[input]{
    \begin{FramePic}[10,10]
   \roi[style=query](2,2)(8,9){$R$}
   \roi[style=query](1,1)(6,6){$S$}
   \roi[style=query](2,4)(9,8){$T$}
 \end{FramePic}
}
\subfigure[input]{
  \begin{FramePic}[10,10]
  \roI[style=frame](0,9)(10,10){$\perp$}{$x_{00}$}

  \roI[style=frame](0,8)(2,9){$\perp$}{$x_{10}$}
   \roI[style=query](2,8)(8,9){$\{R\}$}{$x_{11}$}
  \roI[style=frame](8,8)(10,9){$\perp$}{$x_{12}$}

  \roI[style=frame](0,6)(2,8){$\perp$}{$x_{20}$}
   \roI[style=query](2,6)(8,8){$\{R,T\}$}{$x_{21}$}
   \roI[style=query](8,6)(9,8){$\{T\}$}{$x_{22}$}
  \roI[style=frame](9,6)(10,8){$\perp$}{$x_{23}$}

  \roI[style=frame](0,4)(1,6){$\perp$}{$x_{30}$}
   \roI[style=query](1,4)(2,6){$\{S\}$}{$x_{31}$}
   \roI[style=query](2,4)(6,6){$\{R,S,T\}$}{$x_{32}$}
   \roI[style=query](6,4)(8,6){$\{R,T\}$}{$x_{33}$}
   \roI[style=query](8,4)(9,6){$\{T\}$}{$x_{34}$}
  \roI[style=frame](9,4)(10,6){$\perp$}{$x_{35}$}

  \roI[style=frame](0,2)(1,4){$\perp$}{$x_{40}$}
   \roI[style=query](1,2)(2,4){$\{S\}$}{$x_{41}$}
   \roI[style=query](2,2)(6,4){$\{R,S\}$}{$x_{42}$}
   \roI[style=query](6,2)(8,4){$\{R\}$}{$x_{43}$}
  \roI[style=frame](8,2)(10,4){$\perp$}{$x_{44}$}

  \roI[style=frame](0,1)(1,2){$\perp$}{$x_{50}$}
   \roI[style=query](1,1)(6,2){$\{S\}$}{$x_{51}$}
  \roI[style=frame](6,1)(10,2){$\perp$}{$x_{52}$}

  \roI[style=frame](0,0)(10,1){$\perp$}{$x_{60}$}

  \end{FramePic}
}
  \caption{}
  \label{fig:multi}
\end{figure*} 


Since we have assumed the the input chunks are in integral numbers of
rows, we can define the above formulation in images algebra terms as
follows, where $X$ is an input image:

\begin{align*}
x = & (x_0|x_1|x_2|\ldots|x_n)^T \\
x = & \left( \begin{array}{c}
    x_0 \\
    \hline
    x_1 \\
    \hline
    \ldots \\
    \hline
    x_n
  \end{array}\right)
  &\text{where for each} x_i \\
x_i = & (x_{i0}|x_{i1}|x_{i2}|\ldots|x_{im})
\end{align*}

That is, each image can be divided into a number of row chunks, each
of which can be divided into a number of different column chunks.  If
we define $X_q$ as the set of $x_{ij}$ chunks which satisfy query $q$,
then the resultant image for query $q$ is composed with the equation.

\begin{align*}
x_q =& (q_0|q_1|q_2|\ldots|q_n)^T \\
&\text{where for each } q_i \\
q_i =& (x_{ij}|x_{i,j+1}|x_{i,j+2}|\ldots|x_{i,j+k}), x_{ij} \in Q
\end{align*}

Figure~\ref{fig:mutli}b shows the $x_{ij}$ chunks resulting from the
input queries.  For exmaple, pointset $R$ can be composed with,

\[ R = \left| \begin{array}{c}
    x_{11} \\ 
    \hline 
    x_{21} \\
    \hline
    (x_{32}|x_{33}) \\
    \hline
    (x_{42}|x_{43})
  \end{array}
\right| \]

The output shown in Figure~\ref{fig:multi}b suggests a number of data
strutures, most notably a corner stitched
structure~\cite{ouster:84:corner-stitc}.  However, for $n$ query
pointsets, there are $O(n^2)$ resultant chunks and the size of
structure that enumerates each chunk as in corner-stitching, would
also have a size $O(n^2)$.  However, we can again take advantage of
the fact that the input stream is in row scan order.  If we maintain
the current state of query intersections for the input geostream, then
when a new input chunk comes in we need only check if the new row(s)
are entering or exiting a query pointset.  If not, we rechunk as
before, otherwise we update our rechunking boundaries, by adding or
removing the new query.  The state that we maintain for the current
row is a sorted list of the x-values for each rechunk boundary, and
the set of queries that each new chunk satisfies.

To support this we only need to maintain an in-order list of the
y-values for each query pointsets maximum and minimum.  There is a
small problem with this arrangement.  All geostreams are in row scan
order, but not ncecessarily starting from the maximum y value.  If a
new geostream arrives at a lower value, then we would need to start
from the top and build the starting set of x-values.  In practice, we
will use a skip list~\cite{pugh:90:skip} to maintain our y-values, and
will keep track of starting x-value sets, for the top few levels in
our list.  This will allow use to find the nearest maintained x-value
set and work forwards or backwards from there for the first row in any
new geostream.

\paragraph{Future Directions}

The GeoStream format is currently too simple.  It is not able to
described relaitonships between image streams, nor can it accept
non-homgenous data within the stream.  These are significant
limitations for streaming data sucah as satellitle imagery.  For
example, the GOES satellite, besudes the imager data stream described
above, also has an additional instrument, the sounder.  This data
stream has many more channels that are more closely organized than the
imager data.  We are working to develop a geostream model that can
carry all this information in a single stream.

\bibliographystyle{plain}
\bibliography{2dq.bib}

\paragraph{Appendix A: GOES data stream} 

We will describe the format of one of our main testbed data sources,
National Oceanic and Atmospheric Administration's (NOAA's)
Geostationary Operational Environmental Satellite (GOES)
satellite~\cite{goes}.  All data from the GOES satellite is
transferred via the GVAR data stream, a format specific to NOAA GOES
instruments.  Figure~\ref{fig:gvar} summarizes the GVAR data stream.

\begin{figure*}[htbp]
  \begin{center}
    \includegraphics[width=16cm]{gvar.eps}
    \caption{GOES GVAR data stream}
    \label{fig:gvar}
  \end{center}
\end{figure*}

The GVAR stream is a continuous data stream transmitting at
approximately 2.1 Mb/sec.  It has two instruments, the imager and
sounder, which have 5 and 19 spectral channels respectively.  
The imager scans various sections of the earth's surface, North to
South, about once every fifteen minutes in continuous frames.  A frame
varies in size from about 100MB to 400MB, depending on the size of the
region scanned.  The spatial extent of an imager pixel varies between
spectral channels.  Typically a frame is transmitted completely before
another starts, but priority frames can be inserted into a normal
frame stream.

An individual frame is made up of a large number of scans which are
narrow swathes of data corresponding to the physical sweep of the
instrument sensors from East to West.  The scans themselves are made
up of 11 blocks of data, one documentation block and 10 data blocks
for the pixel data.  The block is the atomic unit of transfer from the
satellite to the receivers.  Documentation blocks are about 64K bits
long, and data blocks can vary from 32K to 229K bits of data, depending
on the width of the scan.  The documentation block includes operational
parameters such as the current location of the satellite, the current
frame and scan, parameters for each instrument, and additional
information for each spectral channel.  The 11 blocks of a scan are
transmitted together in order.  Sounder data, and other auxiliary data
are transmitted between complete scans.
%
Each block includes a 10K bit synchronization tag, which receivers use
to align reception to the blocks.  All blocks have a small common
header.  They include an identifier, the size and type of block, the
block acquisition time and other information.  Each block type must be
read differently, as their format varies considerably.

\end{multicols}

\end{document}

How do we make sure that the metadata from the images is sampled
first, before going thru the cookie cutter.  That is, we wnat to know
that a a subset/etc request is satisfied, BEFORE we try cutting up the
individual chunks.  In out-of-order processing this is hard to do.  I
think that is one reason that I though about moving to the
Query-subset strategy, because if we define supersets along the way as
waypoints for the queries, then we have a reseasonable way to do this,
since the supersets are processed before the subsets.  One problem?
with this is that we need to have many cookie cutting subroutines,
however, maybe that really doesn't increase the time of the query.
