\section{Introduction}
\label{sec:intro}

New methods for processing streaming data \cite{babcoc02model-issues,
  carney02monit-stream, heller00adapt-query} have a great deal of
potential impact for remotely-sensed geospatial image data originating
from the various satellites orbiting the Earth.  Besides its
typically large bandwidth, \acf{RSI} data has a number of
characteristics that are different from generic streaming data.  One
important difference is that streaming RSI data is highly organized
with respect to it's spatial components.  This organization varies for
different data streams, but generally image data will come into the
stream as single objects.  The objects may be individual pixels, rows
of pixels, or complete images.  The organization of the pixels within
these objects is well defined, and consecutive objects are spatially
close to one another.

In addition, most queries to an \ac{RSI} data stream include
operations to restrict the geospatial data to be precessed to
specified regions of interest.  Therefore, an RSI stream management
system needs to efficiently intersect incoming geospatial image
data with a possibly large number of query regions.

In this paper, we present a method for intersection incoming
geospatial image data with multiple spatial restrictions, that is,
queries that request incoming data over particular regions only. For
this, we introduce the \acf{ct}, a structure to index query regions
and to provide for efficient insertions and deletions of queries. In
particular, the DCT supports stabbing point queries. That is, for
incoming geospatial image data, the structure is used to efficiently
determine what queries are affected by the incoming data. For this, we
exploit the trendiness inherent to most types of streaming RSI data.
Based on the information provided by the DCT, query plans can be
generated and incoming data can be pipelined to respective query
operators, thus providing the basis for multiple-query processing models
and concepts for streaming RSI data.

%Treating remotely sensed data as a stream of data is inspired by the
%recent advancements in \acl{dsms}~. In such systems, data
%arrives in multiple, continuous, and time-varying data streams and
%does not take the form of persistent relations.  

%% discuss a method for intersecting an incoming image
%% data stream with multiple spatial restrictions, that is queries that
%% request incoming data streams over a particular area only.  We assume
%% that the incoming data stream is organized in one of the three most
%% common organizations described below and that queries expect their
%% results in the same organization.  In addition, we require a data
%% structure that is dynamically updatable to support queries being
%% inserted and deleted.

The remainder of the paper is structured as follows. In
Section~\ref{sec:model}, we outline the data and query model
underlying \ac{RSI}.  Section~\ref{sec:dct} describes the \ac{ct} in
detail.  Section~\ref{sec:performance} discusses the performance of
the \ac{ct}, and implications of input regions and stabbing point
trends.  Section~\ref{sec:mods} describes extensions and modifications
of the \ac{ct} for similar problems.

%%% Local Variables: 
%%% mode: latex
%%% TeX-master: "hart04stdbm"
%%% End: 
