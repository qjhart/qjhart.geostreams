\documentclass{stdbm}

% This section needs to be included other ways, 
% or used to create eps files of the images.
\usepackage{clrscode}
\usepackage{acronym}
\usepackage{epsfig}
\usepackage{geostreams}

\usepackage{float}
\floatstyle{ruled}
\newfloat{algorithm}{tbph}{aux}
\floatname{algorithm}{Algorithm}

\usepackage{theorem}

{\theorembodyfont{\rmfamily}
\newtheorem{definition}{Definition}[section]
\newtheorem{lemma}{Lemma}[section]
\newtheorem{assumption}[definition]{Assumptions}
\newtheorem{example}[definition]{Example}
\newtheorem{prop}[definition]{Proposition}
\newtheorem{theorem}[definition]{Theorem}
}


\acrodef{NOAA}{National Oceanic and Atmospheric Administration}
\acrodef{GOES}{Geostationary Operational Environmental Satellite}
\acrodef{CIMIS}{California Irrigation Management Information System}

%\acrodef{ct}[\id{DCT}\ ]{Dynamic Cascade Tree}
\acrodef{ct}[DCT]{Dynamic Cascade Tree}
\acrodef{adt}{Abstract Data Type}
\acrodef{RSI}{Remotely-Sensed Imagery}
\acrodef{dsms}{data stream management system}
\acrodef{dbms}{database management system}
\acrodef{ogis}{OpenGIS}
\acrodef{wms} {Web Map Services}
\acrodef{utm}{Universal Transverse Mercator}

\newcommand{\ct}{\id{DCT}}
\newcommand{\List}{\id{List}}
\newcommand{\YList}{\id{2-Key-List}}
%\newcommand{\meth}{\negthickspace \rightarrow \negthickspace}
%\newcommand{\meth}{\negmedspace \centerdot}
\newcommand{\meth}{.}

\newcommand{\inC}[1]{{\id{DCT_{#1}}}}
\newcommand{\X}{{\inC{X}}}
\newcommand{\Y}{{\inC{Y}}}
\newcommand{\A}{{\inC{A}}}
\newcommand{\cn}{{\inC{cn}}}
\newcommand{\cnx}{{\inC{cn_x}}}
\newcommand{\cny}{{\inC{cn_y}}}

\psset{unit=0.2}

\usepackage{pst-3d}

\title{Indexing Query Regions for Streaming Geospatial Data}
%with Dynamic Cascading Trees}

%\author{
%\begin{center}
%  \begin{tabular}{cc@{\hspace*{5ex}}c}
%Quinn Hart & & Michael Gertz\\
%CalSpace & & Department of Computer Science\\
%University of California, Davis & & University of California, Davis\\
%Davis, CA, USA & &Davis, CA, USA\\
%qjhart@ucdavis.edu & & gertz@cs.ucdavis.edu
%  \end{tabular}
%\end{center}
%}
%\institution{~~}
\author{\parbox{6in}{\parbox{2.5in}{\center Quinn Hart} \hfill \parbox{2.5in}{\center Michael Gertz}}}
\institution{
\parbox{6in}{\parbox{2.5in}{\center%
CalSpace \\
University of California, Davis\\
Davis, CA, USA \\
qjhart@ucdavis.edu}
%
\hfill
%
\parbox{2.5in}{\center %
Department of Computer Science \\
University of California, Davis\\
Davis, CA, USA\\
gertz@cs.ucdavis.edu}
}}

\date{25 May 2004}

\begin{document}

\maketitle
\begin{abstract}
  
  The \acf{ct}, a structure designed to index query regions for an
  incoming multi-dimensional data stream, is introduced.  The DCT is
  designed for a stream management system with a particular focus on
  \acf{RSI} data streams.  For these streams, an important query
  operation is to efficiently restrict incoming geospatial data to
  specified regions of interest.  As nearly every query to an RSI
  stream has a spatial restriction, it makes sense to optimize
  specifically for this operation.  In addition, spatial data is
  highly ordered in it's arrival and the \ac{ct} takes advantage of
  this trendiness.  The problem generalizes to solving many
  \emph{stabbing point} queries.  While the worst case performance is
  quite bad, the \ac{ct} performs very well when the stabbing point
  exhibits certain trending characteristics that are common in
  RSI data streams. This paper describes the \ac{ct}, discusses
  performance issues, and provides extensions of the \ac{ct}.
% for other
%  settings.
\end{abstract}

\section{Introduction}
\label{sec:intro}

New methods for processing streaming data \cite{babcoc02model-issues,
  carney02monit-stream, heller00adapt-query} have a great deal of
potential impact for remotely-sensed geospatial image data originating
from the various satellites orbiting the Earth.  Besides its
typically large bandwidth, \acf{RSI} data has a number of
characteristics that are different from generic streaming data.  One
important difference is that streaming RSI data is highly organized
with respect to it's spatial components.  This organization varies for
different data streams, but generally image data will come into the
stream as single objects.  The objects may be individual pixels, rows
of pixels, or complete images.  The organization of the pixels within
these objects is well defined, and consecutive objects are spatially
close to one another.

In addition, most queries to an \ac{RSI} data stream include
operations to restrict the geospatial data to be precessed to
specified regions of interest.  Therefore, an RSI stream management
system needs to efficiently intersect incoming geospatial image
data with a possibly large number of query regions.

In this paper, we present a method for intersection incoming
geospatial image data with multiple spatial restrictions, that is,
queries that request incoming data over particular regions only. For
this, we introduce the \acf{ct}, a structure to index query regions
and to provide for efficient insertions and deletions of queries. In
particular, the DCT supports stabbing point queries. That is, for
incoming geospatial image data, the structure is used to efficiently
determine what queries are affected by the incoming data. For this, we
exploit the trendiness inherent to most types of streaming RSI data.
Based on the information provided by the DCT, query plans can be
generated and incoming data can be pipelined to respective query
operators, thus providing the basis for multiple-query processing models
and concepts for streaming RSI data.

%Treating remotely sensed data as a stream of data is inspired by the
%recent advancements in \acl{dsms}~. In such systems, data
%arrives in multiple, continuous, and time-varying data streams and
%does not take the form of persistent relations.  

%% discuss a method for intersecting an incoming image
%% data stream with multiple spatial restrictions, that is queries that
%% request incoming data streams over a particular area only.  We assume
%% that the incoming data stream is organized in one of the three most
%% common organizations described below and that queries expect their
%% results in the same organization.  In addition, we require a data
%% structure that is dynamically updatable to support queries being
%% inserted and deleted.

The remainder of the paper is structured as follows. In
Section~\ref{sec:model}, we outline the data and query model
underlying \ac{RSI}.  Section~\ref{sec:dct} describes the \ac{ct} in
detail.  Section~\ref{sec:performance} discusses the performance of
the \ac{ct}, and implications of input regions and stabbing point
trends.  Section~\ref{sec:mods} describes extensions and modifications
of the \ac{ct} for similar problems.

%%% Local Variables: 
%%% mode: latex
%%% TeX-master: "hart04stdbm"
%%% End: 

\section{Data and Query Model}
\label{sec:model}

Our data model for RSI data is based on raster images. To allow for
different types of objects called image, we employ some concepts from
the Image Algebra \cite{ritter99image-algeb,wilson01handb-comput} and
extend these concepts to account for the specifics of RSI data.

An image consists of a set of points and values associated with these
points. The \emph{point set} of an image is a set of points and an
associated measure of distance between points, making it a simple
topological space. �As our interest is in RSI, we choose as point set
\ps{X} a subset of $\mathbb{Z}^3$, with a point $\pt{x} \in \ps{X}$ of
the form $\pt{x} = \langle x,y,t \rangle$. The pair $\langle x,y
\rangle$ denotes a spatial location in some absolute coordinate
system, and the component $t$ is a timestamp. Thus, a point set
exhibits spatio-temporal characteristics.
%
A \emph{value set} \vs{V} provides the values associated with points
in a given point set. In the case of color images, $\mathbb{V}$ is a
subset of $\mathbb{Z}^3$ (for red, green, and blue), and for
gray-scale images, it is a subset of \vs{Z}. 
%In general, any value set
%can be chosen for a point set to describe a discrete or physical
%image. 
Based on the concepts of point and value sets, we can now give a
functional representation of an image.

\begin{definition}
 �Given a point set \ps{X} and value set \vs{V}. A \vs{V}-valued
 �image \im{i} is a function from \ps{X} to \vs{V}, denoted $\im{i} =
 �\{(\pt{x}, \im{i}(\pt{x})) \mid \pt{x} \in \ps{X}\}$. The pair
 �$(\pt{x}, \im{i}(\pt{x}))$ is called a \emph{pixel} of \im{i}.
 �\pt{x} is the spatio-temporal component of the pixel and
 �$\im{i}(\pt{x}) \in \vs{V}$ is the \emph{pixel value} at point �\pt{x}.
\end{definition}

In general, there are no restrictions on the shape or order points in
a point set can have. However, we will restrict our discussions to
points and point sets located in an absolute coordinate system that
provides the basis for formulating queries against a stream of RSI
data.

% where the spatial location $\langle 0, 0 \rangle$ corresponds
%to the origin of the coordinate system (i.e., the leftmost lower
%corner).

%Figure \ref{fig:pointset} shows an example of a point set.

Different types of RSI admit different possible orderings and
structures.  Some sensors, such as NOAA's Geostationary Operational
Environmental Satellite (GOES), obtain RSI data image-by-image.
Although conceptually the data collected by GOES can be viewed as a
stream of images, individual images are obtained in a \emph{row-scan
  order} in which pixels are delivered line-by-line (see also
Figure~\ref{fig:pointsets}). Other types of remote sensing equipment
even gathers data on a point-by-point basis.
% in which two consecutive
%points in an RSI stream have a very close spatial and temporal
%proximity.


\newpsobject{cover}{psframe}{linecolor=white,fillstyle=solid,fillcolor=white,linewidth=1pt}
\newpsobject{pointset}{psgrid}{gridcolor=lightgray,subgriddiv=0,gridlabels=0,gridwidth=0.5pt}
\newpsobject{edge}{psframe}{linecolor=black,linewidth=0.5pt}

\begin{figure}[htbp]
 �\centering
\subfigure{
 �\begin{pspicture}(-2.5,0)(9.5,15)
 � �%\psgrid(-3,0)(10,15)
 � �\psset{viewpoint=-0.25 -1 1}
 � �\ThreeDput[normal=0 -1 0](1,9,1){\cover(8,8)\pointset(8,8)\edge(8,8)}
 � �\ThreeDput[normal=0 -1 0](1,6,1){\cover(6,6)\pointset(6,6)\edge(6,6)}
 � �\ThreeDput[normal=0 -1 0](4,3,0){\cover(7,7)\pointset(7,7)\edge(7,7)}
 � �\ThreeDput[normal=0 -1 0](2,0,2){\cover(4,4)\pointset(4,4)\edge(4,4)}
 � �\ThreeDput[normal=0 0 1]{\psline[linewidth=0.3mm]{->}(0,0)(10,0)}
 � �\ThreeDput[normal=0 0 1]{\psline[linewidth=0.3mm]{->}(0,0)(0,10)}
 � �\ThreeDput[normal=1 0 0]{\psline[linewidth=0.3mm]{->}(0,0)(0,10)}
 � �\rput(10,2){\Large x}
 � �\rput(0,9){\Large y}
 � �\rput(-2.5,8){\Large t}
 �\end{pspicture}}
\subfigure{
 �\begin{pspicture}(-2.5,0)(9.5,15)
 � �\psset{viewpoint=-0.25 -1 1}
 � �\ThreeDput[normal=0 -1 0](1,8.5,1){\cover(6,1)\pointset(6,1)\edge(6,1)}
 � �\ThreeDput[normal=0 -1 0](1,8,2){\cover(6,1)\pointset(6,1)\edge(6,1)}
 � �\ThreeDput[normal=0 -1 0](1,7.5,3){\cover(6,1)\pointset(6,1)\edge(6,1)}
 � �\ThreeDput[normal=0 -1 0](1,7,4){\cover(6,1)\pointset(6,1)\edge(6,1)}
 � �\ThreeDput[normal=0 -1 0](1,6.5,5){\cover(6,1)\pointset(6,1)\edge(6,1)}
 � �\ThreeDput[normal=0 -1 0](1,6,6){\cover(6,1)\pointset(6,1)\edge(6,1)}
 � �\ThreeDput[normal=0 -1 0](4,6,0){\cover(7,1)\pointset(7,1)\edge(7,1)}
 � �\ThreeDput[normal=0 -1 0](4,5.5,1){\cover(7,1)\pointset(7,1)\edge(7,1)}
 � �\ThreeDput[normal=0 -1 0](4,5,2){\cover(7,1)\pointset(7,1)\edge(7,1)}
 � �\ThreeDput[normal=0 -1 0](4,4.5,3){\cover(7,1)\pointset(7,1)\edge(7,1)}
 � �\ThreeDput[normal=0 -1 0](4,4,4){\cover(7,1)\pointset(7,1)\edge(7,1)}
 � �\ThreeDput[normal=0 -1 0](4,3.5,5){\cover(7,1)\pointset(7,1)\edge(7,1)}
 � �\ThreeDput[normal=0 -1 0](4,3,6){\cover(7,1)\pointset(7,1)\edge(7,1)}
 � �\ThreeDput[normal=0 -1 0](2,1.5,2){\cover(4,1)\pointset(4,1)\edge(4,1)}
 � �\ThreeDput[normal=0 -1 0](2,1,3){\cover(4,1)\pointset(4,1)\edge(4,1)}
 � �\ThreeDput[normal=0 -1 0](2,.5,4){\cover(4,1)\pointset(4,1)\edge(4,1)}
 � �\ThreeDput[normal=0 -1 0](2,0,5){\cover(4,1)\pointset(4,1)\edge(4,1)}
 � �\ThreeDput[normal=0 0 1]{\psline[linewidth=0.3mm]{->}(0,0)(10,0)}
 � �\ThreeDput[normal=0 0 1]{\psline[linewidth=0.3mm]{->}(0,0)(0,10)}
 � �\ThreeDput[normal=1 0 0]{\psline[linewidth=0.3mm]{->}(0,0)(0,10)}
 � �\rput(10,2){\Large x}
 � �\rput(0,9){\Large y}
 � �\rput(-2.5,8){\Large t}
 �\end{pspicture}
}
\subfigure{
 �\begin{pspicture}(-2.5,0)(9.5,15)
 � �\psset{viewpoint=-0.25 -1 1}
 � �\ThreeDput[normal=0 -1 0](1,7.5,3){\psframe[linewidth=0.5pt,fillstyle=solid,fillcolor=white](1,1)}
 � �\ThreeDput[normal=0 -1 0](1,7,3){\psframe[linewidth=0.5pt,fillstyle=solid,fillcolor=white](1,1)}
 � �\ThreeDput[normal=0 -1 0](2,6.5,3){\psframe[linewidth=0.5pt,fillstyle=solid,fillcolor=white](1,1)}
 � �\ThreeDput[normal=0 -1 0](3,6,3){\psframe[linewidth=0.5pt,fillstyle=solid,fillcolor=white](1,1)}
 � �\ThreeDput[normal=0 -1 0](4,5.5,3){\psframe[linewidth=0.5pt,fillstyle=solid,fillcolor=white](1,1)}
 � �\ThreeDput[normal=0 -1 0](5,5,3){\psframe[linewidth=0.5pt,fillstyle=solid,fillcolor=white](1,1)}
 � �\ThreeDput[normal=0 -1 0](6,4.5,3){\psframe[linewidth=0.5pt,fillstyle=solid,fillcolor=white](1,1)}
 � �\ThreeDput[normal=0 -1 0](5,4,4){\psframe[linewidth=0.5pt,fillstyle=solid,fillcolor=white](1,1)}
 � �\ThreeDput[normal=0 -1 0](4,3.5,4){\psframe[linewidth=0.5pt,fillstyle=solid,fillcolor=white](1,1)}
 � �\ThreeDput[normal=0 -1 0](3,3,4){\psframe[linewidth=0.5pt,fillstyle=solid,fillcolor=white](1,1)}
 � �\ThreeDput[normal=0 -1 0](2,2.5,4){\psframe[linewidth=0.5pt,fillstyle=solid,fillcolor=white](1,1)}
 � �\ThreeDput[normal=0 -1 0](2,2,4){\psframe[linewidth=0.5pt,fillstyle=solid,fillcolor=white](1,1)}
 � �\ThreeDput[normal=0 -1 0](2,1.5,5){\psframe[linewidth=0.5pt,fillstyle=solid,fillcolor=white](1,1)}
 � �\ThreeDput[normal=0 -1 0](3,1,5){\psframe[linewidth=0.5pt,fillstyle=solid,fillcolor=white](1,1)}
 � �\ThreeDput[normal=0 -1 0](4,.5,5){\psframe[linewidth=0.5pt,fillstyle=solid,fillcolor=white](1,1)}
 � �\ThreeDput[normal=0 -1 0](5,0,5){\psframe[linewidth=0.5pt,fillstyle=solid,fillcolor=white](1,1)}
%
 � �\ThreeDput[normal=0 0 1]{\psline[linewidth=0.3mm]{->}(0,0)(10,0)}
 � �\ThreeDput[normal=0 0 1]{\psline[linewidth=0.3mm]{->}(0,0)(0,10)}
 � �\ThreeDput[normal=1 0 0]{\psline[linewidth=0.3mm]{->}(0,0)(0,10)}
 � �\rput(10,2){\Large x}
 � �\rput(0,9){\Large y}
 � �\rput(-2.5,8){\Large t}
 �\end{pspicture}
}
\caption{%
%
  �Examples of different point set orderings for streaming RSI data:
  image-by-image, line-by-line, and �point-by-point.
%
}
\label{fig:pointsets}
\end{figure}


All the above scenarios describing how RSI data is obtained show an
important characteristics we aim to exploit in our approach:
% to indexing a stream of geospatial image data: 
\emph{the points in a point set exhibit certain trends}. That is,
consecutive points in a stream of RSI data have a close spatial and
temporal proximity. The only exception is where the last point of a
line in an image is followed by the first point of a new image
(scenario on the left in Figure \ref{fig:pointsets}). As we will show
in the following section, knowing about the \emph{trendiness} of
incoming geospatial point data can have a significant influence on how
queries against a stream of RSI data are processed.

Queries against a stream of RSI data are typically continuous queries
that run for a long time and may include complex operators, such as
spatial transformations, sensor-band algebra operations, or aggregates
(see, e.g., \cite{ritter99image-algeb,wilson01handb-comput}, for
operations on point sets). However, since not all applications are
interested in the complete region covered by a sensor (or satellite),
\emph{spatial restriction} is a type of operation common to all
queries. A spatial restriction specifies a \emph{region of interest},
primarily in the form of a rectangle, and typically preceed other
operations on point data. In the following, we describe an index
structure and algorithms to efficiently determine what query regions
are affected by incoming geospatial image data.


%Depending on the geospatial ordering of points in an image, it can be
%chunked in various ways as it's input to operators 

%into the \ac{dsms}. �One way is
%to chunk each pixel separately; that is each tuple is the point and
%value $(\pt(x),a(\pt{x}))$. �For non-homogeneous images this is
%basically the only possible method. �For $n$ rectangular queries, a
%stabbing data structure like an interval tree could choose the
%matching restrictions in $O(\card{a}\lg{n})$ for all $n$ queries, with
%an index structure of $O(n\lg{n})$ size~\cite{spatialalgos}.

%Advantages of representing images in the \ac{dsms} as a set of points
%is that it's the most general; it requires no order on the images; and
%tuples either match restriction queries or not, they do not need to be
%modified. �The main disadvantages are that there is a large overhead
%in representing the point set, and that satisfying restrictions is
%slow.

%Once an order has been described for an image point set other possible
%chunking methods are possible. �For example, for each point in the
%temporal dimension an entire 2-d image could be passed,
%$(x_n,x_x,t,\im{a})$, where the minimum and maximum points, $x_n$, and
%$x_x$ describe the 2-d image point set. �The advantages of this
%formulation is the compact representation of the point set and image.
%the main disadvantage is that for queries that overlap, you get an
%increase in the number and size of tuples in the \ac{dsms} which have
%many redundant pixels.

%For row scan images, a convenient chunking is on each input row. �Each
%new chunk corresponds to discontinuities in the original point set.
%With this chunking, the resultant images can be formed by simply
%concatenating all processed tuples together.

%Chunking by rows has some of the same problems as chunking by 2-d
%images, in that multiple queries can replicate much of the data, and
%the time in creating new tuples can be considerable. �However, for
%normal representations of images, some of this cost can be eliminated
%by retaining the original input row tuple and reusing that image data.
%The key is that no discontinuities exist in the row of image data.
%New restricted rows can use pointers into the existing image data, and
%only need to create new point sets. �Figure~\ref{fig:rows} illustrates
%multiple restrictions using common image data.

%\begin{figure}[htbp]
% �\centering
% �\subfigure[input row]{
% � �\begin{FramePic}[10,10]
% � �\psframe[fillstyle=solid,fillcolor=gray](0,4)(10,5)
% � �\psgrid[gridcolor=lightgray,subgriddiv=0,gridlabels=0,gridwidth=1pt](0,4)(10,5)
% � \roi[style=query](1,1)(6,6){$S$}
% � \roi[style=query](2,4)(9,8){$T$}
% \end{FramePic}}
%\quad
% \subfigure[output tuples]{
% � \begin{pspicture}(10,10)
% � � \uput{7pt}[u](5,6){$S:(s_n,s_x,r,i_s)$}
% � � \psframe[fillstyle=solid,fillcolor=gray](1,6)(6,7)
% � �\psgrid[gridcolor=lightgray,subgriddiv=0,gridlabels=0,gridwidth=1pt](0,6)(10,7)
% � �\uput{7pt}[u](5,2){$T:(t_n,t_x,r,i_t)$}
% � �\psframe[fillstyle=solid,fillcolor=gray](2,2)(9,3)
% � �\psgrid[gridcolor=lightgray,subgriddiv=0,gridlabels=0,gridwidth=1pt](0,2)(10,3)
% � \end{pspicture}
% }
% \caption{%
% � Multiple restriction queries on row chunked images. �By retaining
% � the original image data, all matching restrictions can point into
% � the same image data object. �Only point set information needs to be
% � created for each new tuple.}
% \label{fig:chunk}
%\end{figure}

%%% Local Variables: 
%%% mode: latex
%%% TeX-master: "hart04stdbm"
%%% End: 


\vspace*{-1ex}
\section{The Dynamic Cascade Tree (DCT)}
\label{sec:dct}

The problem with quickly answering multiple queries on a stream of RSI
data is basically solving a normal \emph{stabbing
  query}~\cite{berg00comput-geomet} for a point, the incoming data
stream, and a set of query regions. Many data structures have been
developed for one and two dimensional stabbing queries including,
among others, interval trees, priority search trees, and segment
trees~\cite{berg00comput-geomet}.  For \ac{RSI}, the stabbing points
are special in that the next stabbing point is typically very close to
the previous stabbing point.  The goal is to take advantage of the
trendiness of stabbing points to develop index structures that improve
search performance for subsequent stabbing queries.

The two most common methods for solving stabbing queries in two
dimensions are multi-level segment trees~\cite{berg00comput-geomet}
and R-trees~\cite{guttm84r-tree}.  Using multi-level segment trees,
one dimension of the region is stored in a segment tree, while the
second dimension is indexed with an associated interval structure for
each node in the first segment tree.  Storage for these structures can
be $O(n\lg{n})$, with stabbing query times of $O(\lg^2{n})$.  Dynamic
maintenance of such a structure is more complicated, and requires
larger storage costs~\cite{krevel88concat-segmen}.  It is difficult to
modify the multi-level segment tree to improve results for trending
data.  If the input stabbing point moves a small distance, which
doesn't change the query results, it still would take $O(\lg{n})$ time
to respond.  That is because even if every node in the multi-level
segment tree maintains knowledge of the previous point, it would still
take $\lg({n})$ time to traverse the primary segment tree to discover
that no changes to the query occurred.

R-trees solve the stabbing query problem by recursively traversing
through successive minimum bounding rectangles that include the extent
of all regions in the sub-tree, generally with good performance.
Since these rectangle regions can overlap, there can be no savings
from knowing the previous stabbing query, as there is no way to know
if an entirely new path through the segment tree needs to be
traversed.  R+-trees\cite{sellis87r-tree} can have better performance
for these trending stabbing points, since the minimum bounding
rectangles are not allowed to overlap and so maintaining the previous
query can help verify a query hasn't left a particular region.
R+-trees have problems with redundant storage, dynamic updates, and
potential deadlocks~\cite{manol03r-have}.

A method of dynamically describing a neighborhood of validity for a
stabbing query was proposed in~\cite{zhang03locat}, using R-trees.
This method builds a region of validity around a current point, which
can then be used to verify that a new point will not result in a
different response.  The technique makes a number of additional
queries to the R-tree index in order to build this region.  Like the
technique described below, this could result in a cost savings if many
subsequent stabbing queries are located with the region of validity.

The structure proposed in the following builds an index that is
dynamically tuned to the current location of RSI data (points).  The
idea is to develop a structure that, for a given point, maintains the
regions around that point where the query result will change.
Stabbing queries can determine in constant time if the new stabbing
point has the same result as the previous query and will incrementally
update a new result set based on the previous set when the result is
different.  The structure is designed to be small and quickly allow
for insertions and deletions of new query regions.  It assumes some
particular characteristics from the input stream, notably that the
stream changes in a way that many subsequent point data will
contribute to the same result set(s) to region queries so that the
cost of maintaining a dynamic structure can be amortized over a large
set of queries.  Section~\ref{sec:performance} describes in more
detail the performance implications of the regions and input data
stream.

\vspace*{-1.5ex}
\subsection{DCT Details}

Figure~\ref{fig:cascade-tree} shows a complete overview of the data
structure employed, which we term a \emph{\acf{ct}}.  The figure shows
a set of query regions, the most recent stabbing point and the
associated structures for the \ct.

\begin{figure*}[htbp]
  \centering
  \input{dct.pstex_t}
  \caption{The Dynamic Cascade Tree (DCT)}
  \label{fig:cascade-tree}
\end{figure*}

These associated structures of \ct\ are pleasantly simple extensions
to a binary tree.  In the example and following pseudo-code, we assume
that we have two simple search structures, \List\ and \YList.  \List\ 
supports \proc{Insert(key,value)}, \proc{Delete(key)}, and
\proc{Enumerate}().  Keys in \List\ are unique for each value.  In our
application, we use a simple skip list~\cite{pugh90skip} to implement
\List.  The \YList\ is incrementally more complex.  It supports
\proc{Insert}$(\id{key_1},\id{key_2},\id{value})$,
\proc{Delete}$(\id{key_1},\id{key_2})$ and
\proc{Enumerate}$(\id{key_1})$ using two keys.  The combination of the
two keys are unique to each value.  \proc{Enumerate}$(\id{key_1})$\ 
enumerates all the values in the \YList, entered with \id{key_1}.  An
implementation of \YList\ could be a skip list using \id{key_1}, where
each node has an associated skip list using \id{key_2}.  With this
implementation, for $\YList \meth
\proc{delete}(\id{key_1},\id{key_2})$, if the deletion causes an empty
set in the associated \id{key_1} node,
then that entire node is deleted.%
\footnote{The structures \List\ and \YList\ are defined as above to
  help conceptualize \ct.  In reality, with little modification to the
  algorithms presented, the \YList\ can be implemented as a normal
  \List, simply by using as a key some combination of \id{key_1} and
  \id{key_2} that still groups all \id{key_1}'s together, for example
  $\id{key} \gets \id{key_1} <<
  \func{int}(\lg{\proc{max}(\id{key_2})}+1) + \id{key_2}$.  In order
  to enumerate all regions at a boundary crossing, the algorithms would
  walk the \YList\ comparing right shifted keys that ignore changes in
  \id{key_2}.  This structure could be simpler to implement and does
  not affect the performance of the \ct. }

\List\ leaf nodes contain the $key(s)$ and a pointer to the \id{value}.
Leaf nodes of a \YList\ also have pointers to the next and previous
nodes in sorted order, adding linked list traversal to the leaf nodes.
One reason for choosing a skip list implementation is that the forward
pointers already exist, and only an additional back pointer is added to
a normal skip list.

\ct\ maintains three structures, \Y,\X, and \A, and two pointers to
the current nodes in \Y, and \X, denoted \cnx\ and \cny, respectively.
In addition, identifiers are assigned to each query region inserted
into \ct.

\Y\ is a \YList\ that contains all the $y$ endpoints for every current
region.  \id{key_1}\ is the $y$-value of the endpoint, and \id{key_2}\ 
is the region identifier, \id{r_{id}}.  Since multiple regions can
have the same endpoints, at each leaf node an additional \List\ 
maintains the regions with that endpoint, keyed with \id{r_{id}}.

\X\ is another \YList, with keys on the endpoints $x$-values, and
\id{r_{id}}.  \X\ does not contain the endpoints of all the regions in
\ct, but only of the regions that have $y$ domains that contain the
value of the current $y$-node, \cny.  Again, the leaf nodes of \X\ 
contain \List\ structures that hold multiple regions with the same
endpoint.

\cny\ and \cnx\ are simply pointers to the nodes within both \Y\ and
\X\ corresponding to the location of the most recent stabbing point.
The leaf nodes of \X\ and \Y\ correspond to half-open line segments of
those domains ranging from $[key_i,key_{i+1})$, for any node $i$, in
either \X\ or \Y.

\A\ is the final \List, which contains all the currently selected
regions.  Just as \X\ contains only subsets of the regions of \Y\ that
contain the \cny\ node, \A\ contains the subset of \X\ where the \cnx\ 
node is contained by the $x$ domains of each region.  The \Y, \X, and
\A\ make up a cascade of indexes, each a subset of the previous index
structure.

\vspace*{-1ex}
\subsection{Updating query regions in the DCT}

Initializing the \ct, given a starting stabbing point, simply
initializes the structures, and adds dummy points for \X\ and \Y\ 
outside their valid range, to assign \cnx\ and \cny.
%% Figure~\ref{fig:initialize} shows the pseudo-code for initializing
%% \ct, which creates the internal data structures and two dummy nodes
%% outside the valid range of the regions.  This allows for \cnx, and
%% \cny\ to be defined.

%% \begin{figure}[htbp]
%%   \centering
%% \begin{codebox}
%% \Procname{$\proc{Initialize}(\id{np})$}
%% \li \Comment Input: Current point, \id{np}.
%% \li \Comment Ouput: New search structure, \id{XY\A}.
%% \li $\Y \gets \YList \meth \proc{new}()$
%% \li $\id{y_{node}} \gets \Y \meth \proc{insert}(-\infty,\const{nil})$
%% \li $\X \gets \YList \meth \proc{new}()$
%% \li $\id{x_{node}} \gets \X \meth \proc{insert}(-\infty,\const{nil})$
%% \li $\A \gets \List \meth \proc{new}()$
%% \li $\cn \gets [\id{y_{node}},\id{x_{node}}]$
%% \li \Return \id{DCT}
%% \End
%% \end{codebox}
%% \caption{Initialization of \ac{ct}}
%% \label{fig:initialize}
%% \end{figure}

Algorithm 1 shows the pseudo-code inserting query regions into \ct.
Insertion and deletion are simple routines, where regions are first
inserted/deleted into \Y, and then successively into \X\ and \A, if
the range of the regions are appropriate.  \proc{Delete-Region} is
similar to the insertion. \proc{Delete-Region}\ takes as input a
region.  For deleting a region \id{r} by it's identifier
$\id{r}_{id}$, an additional structure on all region identifiers is
required, as \A\ only maintains pointers to the regions in the
currently active query.

\begin{algorithm}[htbp]
\label{al:insert}
\caption{Inserting Query Regions in \ct}
\begin{codebox}
\Procname{$\proc{Insert-Region}(\id{DCT},\id{r})$}
\li \Comment Input: \id{DCT}, region \id{r}.
\li \Comment Ouput: \id{r} is inserted into \id{DCT}.
\li $\Y \meth \proc{insert}(\id{r}_{y_n},\id{r}_{id},\id{r})$
\li $\Y \meth \proc{insert}(\id{r}_{y_x},\id{r}_{id},\id{r})$
\li \If $(\id{r}_{y_n} <= \cny_{key}$ and $\id{r}_{y_x} > \cny_{key})$
\li   \Then $\X \meth \proc{insert}(\id{r}_{x_n},\id{r}_{id},\id{r})$
\li   $\X \meth \proc{insert}(\id{r}_{x_x},\id{r}_{id},\id{r})$
\li   \If $(\id{r}_{x_n} <= \cnx_{key}$\ \kw{and} 
\zi       $\quad\quad\id{r}_{x_x} > \cny_{key})$
\li     \Then $\A \meth \proc{insert}(\id{r}_{id},\id{r})$
     \End
   \End
\End
\end{codebox}
\end{algorithm}

It should be clear that the structures \X\ and \A\ need to be
maintained both when new regions are inserted and deleted, but also
for each stabbing point.  Since \X\ only contains regions with
overlapping $y$ domains, when a new stabbing point arrives where a $y$
boundary for any region in the \ct\ is crossed, then the \X\ structure
needs to be modified to account for the regions to be included or
deleted from consideration.  A similar method needs to be associated
with boundary crossings in the $x$ dimension, while traversing \X\ and
modifying \A.

\vspace*{-1ex}
\subsection{Querying the DCT}

The algorithm for reporting resultant (active) query regions for a new
stabbing point, \id{np}, begins by traversing the \YList\ \Y\ in the
$y$ direction from the current node, \cny\ to the node containing
$\id{np_y}$ going through every intermediate node using the linked
list access on the leaf nodes of \Y.  At each boundary crossing, as
regions are entered or exited, those regions need to be added into the
\X\ \YList. When the point has traversed to the node containing
$\id{np_y}$, then traversal begins in the $x$ direction, moving from
\cnx to the node containing $\id{np_x}$.  As with \Y, when the
traversal hits $x$ boundary points, then the entered query regions are
added into \A\ and the exited regions are deleted from \A.  When the
traversal reaches $\id{np}$, then \cn\ contains pointers to the nodes
containing the \id{np}, \X\ contains $x$ endpoints to all the regions
with $y$ domains that encompass $\id{np_y}$, and \A\ contains all
regions that contain \id{np}.  \A\ is then enumerated to report all
the query regions that are affected by the new stabbing point, \id{np}.

\begin{figure*}[htbp]
  \centering
  \input{move.pstex_t}
  \caption{Stabbing point moving in \proc{Report-Regions}}
  \label{fig:update}
\end{figure*}

Figure~\ref{fig:update} shows an example of an update of the
structures within \ct, on reporting regions for a new stabbing point.
This extends the example of Figure~\ref{fig:cascade-tree}.  In this
example, the new point has crossed a $y$ boundary that contains two
region endpoints, $c$ and $f$.  As the point is traversed in the $y$
direction to this new point, the $x$ endpoints of region $c$ are
removed from \X, and the endpoints of $f$ are added to \X.  When the
endpoints of these regions are deleted, the regions themselves are
also deleted from \A, and similarly for insertions.  In the example,
$c$ is deleted and $f$ inserted into \A.  After reaching \id{np_y},
\X\ is traversed in the $x$ direction.  In the example, this results
in $e$ being deleted from \A.  Finally, \A\ is enumerated, completing
the procedure.


Algorithm~2 describes the \proc{Report-Regions}\ procedure, which
reports query regions for a new stabbing point, while updating the dynamic
structures of the \ct.  Lines~\ref{li:report-y-loop-begin}
to~\ref{li:report-y-loop-end} update the \X\ and \A\ structures due to
boundary crossings in the $y$ dimension.  Only one of the \kw{while}\ 
loops are executed at each invocation.
Lines~\ref{li:report-x-loop-begin} to~\ref{li:report-x-loop-end}
perform a similar function updating the \A\ structure as the point is
traversed to \id{np_x}.  After traversing to \id{np}, the updated set
of matching regions are contained in \A, and line
\ref{li:report-return} reports those regions.
%

\vspace*{-1ex}
\section{DCT Performance}
\label{sec:performance}
%
The performance of \proc{Report-Regions}\ is highly dependant on the
location of the regions, the trending properties of the stream data,
and the interaction of the two parameters.  For $m$ executions of
\proc{Report-Regions}, with $k$ being the average number of resultant
regions, the average time of execution can range from $O(k)$ in the
best case to $O(n+k\lg{k})$ in the worst case. % (with a modification
%described below, see Figure~\ref{al:mod}.  
Reasonable experiments could be designed that would approach either of
these limits.  Instead, there are rules to consider for the
application of the \ct.

\subsection{Insertions and deletions of query regions}
%
The \id{DCT}\ data structure is generally robust to many insertions
and deletions of regions of interest.  Both items take $O(\lg{n})$
time as the region is potentially added to structures \X, \Y, and \A.
\List\ and \YList\ are simple to maintain dynamically in $O(n)$ space.

\vspace*{-1ex}
\subsection{Number of boundaries crossed}
%
The \ct\ is designed for trending data, which can most quantitatively
be measured with the number of region boundary crossings from one
stream data point to the next. This structure works best when the
number of query boundaries crossed on subsequent input points is not
large.  When no boundaries are crossed, then no internal lists are
modified, and \proc{Report-Regions} runs in $O(k)$ time.  When a
single boundary is crossed, then each region in the crossed $y$ node
needs to be inserted or deleted from the \X\ structure.  This is true
\begin{algorithm}[htbp]
\label{al:report}
\caption{Stabbing queries in \ct}
\begin{codebox}
\Procname{$\proc{Report-Regions(\id{DCT},\id{np})}$}
\li \Comment Input: \id{DCT}, new stabbing point, \id{np}.
\li \Comment Output: List of query regions containing \id{np}.
\li \Comment Walk to new $y$ value, adding/deleting $x$-ranges
\li \While $(\id{np}_y < \cny \meth \id{key})$ \label{li:report-y-loop-begin}
\li   \Do \kw{for} $\id{r} \in \Y \meth \proc{enumerate}(\cny)$
\li   \If $\id{r}_{x_n} = \cny \meth \id{key}$
\li            \Then $\X \meth \proc{delete}(\id{r}_{x_n},\id{r}_{id})$
\li               $\X \meth \proc{delete}(\id{r}_{x_x},\id{r}_{id})$
\li               $\A \meth \proc{delete}(\id{r}_{id})$
\li            \Else $\X \meth \proc{insert}(\id{r}_{x_n},\id{r}_{id},\id{r})$
\li               $\X \meth \proc{insert}(\id{r}_{x_x},\id{r}_{id}),\id{r})$
\li               \If $(\id{r}_{x_n} <= \id{np_x} and \id{r}_{x_x} > \id{np_x})$
\li                  \Then $\A \meth \proc{Insert}(\id{r_{id}},\id{r})$
                     \End
      \End
\li   $\cny \gets \cny \meth \proc{prev}$
  \End
\li \While $(\id{np}_y > \cny \meth \proc{next} \meth \id{key})$
\li   \Do $\cny \gets \cny \meth \proc{next}$
\li     \kw{for} $r \in \Y \meth \proc{enumerate}(\cny)$
\li       \If $r_{y_x} = \cny \meth \id{key}$
\li            \Then $\X \meth \proc{delete}(\id{r}_{x_n},\id{r}_{id})$
\li               $\X \meth \proc{delete}(\id{r}_{x_x},\id{r}_{id})$
\li               $\A \meth \proc{delete}(\id{r}_{id})$
\li            \Else $\X \meth \proc{insert}(\id{r}_{x_n},\id{r}_{id},\id{r})$
\li               $\X \meth \proc{insert}(\id{r}_{x_x},\id{r}_{id},\id{r})$
\li               \If $(\id{r}_{x_n} <= \id{np_x} and \id{r}_{x_x} > \id{np_x})$
\li                  \Then $\A \meth \proc{Insert}(\id{r_{id}},\id{r})$
                     \End
       \End
   \End                                     \label{li:report-y-loop-end}
\li \Comment Walk to new $x$ value, modifying active queries
\li \While $(\id{np}_x < \cnx \meth \id{key})$ \label{li:report-x-loop-begin}
\li   \Do \kw{for} $\id{r} \in \X \meth \proc{enumerate}(\cnx)$
\li       \If $r_{x_n} = \cny \meth \id{key}$
\li            \Then $\A \meth \proc{delete}(\id{r}_{id})$
\li            \Else $\A \meth \proc{insert}(\id{r}_{id})$
            \End
\li     $\cnx \gets \cnx \meth \proc{prev}$ \label{li:report-break}
   \End
%%  \end{codebox}
%% \end{algorithm}

%% \begin{algorithm}
%% \begin{codebox}
%% \setcounter{codelinenumber}{\ref{li:report-break}}
%% %\setlinenumber{li:report-break}
\li \While $(\id{np}_x > \cnx \meth \proc{next} \meth \id{key})$
\li   \Do $\cnx \gets \cnx \meth \proc{next}$
\li     \kw{for} $r \in \X \meth \proc{enumerate}(\cnx)$
\li       \If $r_{x_x} = \cnx \meth \id{key}$
\li            \Then $\A \meth \proc{delete}(\id{r}_{id})$
\li            \Else $\A \meth \proc{insert}(\id{r}_{id})$
            \End
   \End \label{li:report-x-loop-end}
\li \Return $\A \meth \proc{enumerate}$ \label{li:report-return}
\End
\end{codebox}  
\end{algorithm}
for regions whose $x$ domains do not overlap the new stabbing point
\id{np}, and cannot contribute the \A\ structure.  The cost of
\proc{Report-Regions} in this case can be as high as $O(n\lg{n} + k)$,
since many non-overlapping regions are inserted into the \X\ 
structure.  There is no performance difference in whether the crossing
occurs at a single boundary with many regions at that node, or over
many boundary crossings with few regions.

At least one boundary crossing is necessary to fully index a region.
The \id{DCT}\ data structure indexes lazily in the sense that on
insertions of new regions, it does not index on the $x$ values of a
region, so boundary crossing costs must take some time to index on the
$x$ values of regions.  The problem with the \ct\ is that these costs
can occur many times in the travel of the input stabbing points.
Rather than indexing these values once, the \ct\ re-indexes a subset of
points multiple times as boundaries are crossed.  The hope is that
once in a new set of regions many subsequent stabbing points will be
in that area and the low cost of those stabbing points will make up
for the extra costs of maintaining a dynamic index.

\vspace*{-1ex}
\subsection{Trajectory of the trending data point}
%
Another aspect affecting performance is the trajectory of the input
stabbing points.  For example, consider input streams with
trajectories that are increasing or decreasing monotonically in $x$
and $y$.  In these cases, regions are put into the \X\ structure
at most one time.  This means that the total time maintaining the \X\ 
structure is at most $O(n\lg{n})$, which means that the dynamic
maintenance of the \X\ structure takes no more time than static
indexing methods would require.  The total cost of $m$ stabbing
queries over that trajectory would be $O(n\lg{n}+mk)$ where $k$ is the
average number of regions per stabbing point.  For a segment tree
implementation, the total cost would be $O(n\lg{n}+m\lg^2{n}+mk)$,
which includes the static cost of maintaining a segment tree and does
not include extra costs for dynamically maintaining that  tree.

On the other hand, data with a more erratic trajectory can result in
poor performance.  Consider a point that repeatedly crosses a single
boundary containing all $n$ region boundaries.  Again, each iteration
would require $O(n\lg{n})$ time, as the \X\ and \A\ structures are
both repeatedly made up and torn down.

Also, the \ct\ as described in the example above favors stream data
points that trend in the $x$ direction over data that trends in the
$y$ direction.  The reason for this is basically that more regions
added into the \X\ structure are ended up being reported, and the
dynamic structure building is not wasted.  Also, there are fewer
insertions and deletions in the \X\ structure in the first place.
When the stabbing point \id{np} crosses a boundary in the $x$
direction, it still takes $O(\lg{n})$ time to satisfy, as it updates
\A, but this is more useful work in updating \A\ than a $y$ crossing
boundary, which can spend wasteful time adding points into \X\ that
might never be used.  Where a $y$ trending stabbing trajectory can
have a worst case time in \proc{Report-Regions}\ of $O(n\lg{n}+k)$,
the worst case time in $x$ trending stabbing points is $O(n+k)$, when
worthless insertions into \A\ are skipped as described below.

This shows that order in the cascade is very important, and dimensions
that see more boundary crossings for subsequent stabs into the \ct\
should be pushed deeper into the structure.  Boundary crossings are of
course dependant on the trajectory of the stabbing point and the
organization of the regions in the \ct.

\vspace*{-1ex}
\subsection{Skipping worthless insertions}
%
As mentioned, when the next point of the input stream trends a long
way with respect to the number of query boundaries traversed, then the
time for \proc{Report-Regions} goes up to at least the number of
regions contained in all the boundaries crossed.  Regions that are
both entered and exited in the course of a single traversal to
\id{np} are even worse.  Their endpoints are needlessly added, then
deleted from \X, at a cost of up to $O(\lg{n})$, and never queried.
This can easily be remedied, but for clarity was left out of the
initial \proc{Report-Regions} algorithm.  When encountering a region
at a $y$ boundary crossing, simply check that it will remain a valid
region when \id{np} has finished it's traverse before inserting into
the \X\ structure.  This prevents wasted index modifications, but
does not help with the basic problem of long traverses of \id{np}\ or
input points that cross back and forth across expensive
$y$-boundaries.  Algorithm~3 shows the modifications made to
\proc{Report-Regions} in lines~\ref{li:report-y-loop-begin} to
\ref{li:report-y-loop-end} in Algorithm~2.

\begin{algorithm}[htbp]
\label{al:mod}
\caption{Stabbing query modification}
\begin{codebox}
\setcounter{codelinenumber}{\ref{li:report-y-loop-begin}}
%\setlinenumber{li:report-y-loop-begin}
\li \While $(\id{np}_y < \cny \meth \id{key})$
\li   \Do \kw{for} $\id{r} \in \Y \meth \proc{enumerate}(\cny)$
\li       \If $(\id{r}_{x_n} = \cny \meth \id{key})$
\li            \Then \If $(r_{x_x} < \id{np}_y)$
\li                 \Then $\X \meth \proc{delete}(\id{r}_{x_n},\id{r}_{id})$
\li                   $\X \meth \proc{delete}(\id{r}_{x_x},\id{r}_{id})$
\li               $\A \meth \proc{delete}(\id{r}_{id})$
\li                  \Else $\X \meth \proc{insert}(\id{r}_{x_n},\id{r}_{id})$
\li                   $\X \meth \proc{insert}(\id{r}_{x_x},\id{r}_{id})$
                  \End
       \End
\li     $\cny \gets \cny \meth \proc{prev}$
   \End
\end{codebox}
\end{algorithm}

\vspace*{-1ex}
\section{Extensions and Modifications to the DCT}
\label{sec:mods}

In Section~\ref{sec:dct}, discussion centered on answering a simple
stabbing query for a single point and a number of regions in a two
dimensional space.  This basic framework can undergo some simple
modifications to handle a number of similar types of queries.

\vspace*{-1ex}
\subsection{Non-point stabbing queries}
%
The first general area is for regions other than points.  The stabbing
query can be changed from a single point to a constant size rectangle
very simply.  In this case, track the center location of the stabbing
rectangle, and when inserting new regions of interest, extend the
boundaries by half the width and height of the rectangle queries.
Intersections of the modified regions and the stabbing point will
coincide with intersections of the original regions and the rectangle
query.  For more dynamic stabbing rectangles, track both edges of the
stabbing rectangle in both the \Y\ and \X\ structures.  The leading
edge of the lines in $y$ will track insertions into \A\ and the
trailing edges will track deletions.  Leading and trailing are with
respect to the previous stabbing rectangle.  Rectangles that are
growing in size from the previous rectangle may have two leading
edges, and shrinking rectangles will have two trailing edges.  A
similar strategy is used for mapping the \X\ structure to the \A\ 
structure.  This does not affect the time complexity of the
\proc{Report-Regions}\ algorithm, or the size of \ct.

Many remote sensing image data comes in a row-by-row scheme.  For that
special case of a general stabbing rectangle, use and maintenance of
the \Y\ structure remains as in the stabbing point example, and only
the \X\ structure is modified for different lengths of the individual
rows of data.

\vspace*{-1ex}
\subsection{Adding dimensions}
%
Adding an additional dimension is a simple extension by adding another
intermediate layer to the \X, \Y, \A\ cascade.  For example, a time
dimension on a rectangle query, or in this instance a cube query,
could be added.  Dimensions that run into infinity, which is common
for temporal queries, are not more complex in this structure, as they
can be in other representations.  As discussed in
Section~\ref{sec:performance}, it is best to order the dimensions so
the most varying is on the deeper levels of the \ct.  The monotonic
increase of time would make it a good candidate for the level before
the \A\ structure.  However, if a systems contained regions with
mostly unbounded temporal queries, then it could also be located at
the first level.  One nice feature with making time the first
structure of the cascade is that it also does double duty in providing
a structure to prune regions that have expired.  For a geospatial
example, if incoming pixels are timestamped, then new stabbing points
that have identified regions whose time extent has ended can be
removed from the time structure as well as the other cascaded
structures.

%% For real-time streaming queries, a time structure can be implemented
%% as e.g., a priority queue rather than as a \YList, since next stabbing
%% points cannot go back in time.

%% \subsection{ Multiple streaming locations}
%% %
%% The discussion in Section~\ref{sec:dct} considers only a
%% single input stabbing stream.  It Is possible to have more than one
%% input point, but since structures \X\ and \A\ are both dynamically
%% built with respect to current stabbing point, then 

\vspace*{-1ex}
\subsection{Speed over Size}
%
In Section~\ref{sec:performance} the poor performance of this
structure for certain trending data was discussed.  In particular, when
data points move back and forth among a large number of regions without
staying in those regions for any length of time, then there is a large
cost associated with building up and tearing down the \X\ structure,
without any amortization of that cost over a larger number of queries.
If the size of the data structure is of less importance, then the time
spent building up these intermediate structures can be done once and
saved for subsequent queries.  The idea is that instead of having one
\X\ structure, one has a \X\ for each boundary in the \Y\ structure.
This increases the size of the structure to $O(n^2)$, but all
subsequent queries can be answered without rebuilding.  

To continue to tweak the data structure for this example, since a
different \X\ structure is stored in each $y$ node, it is no longer
required that the \Y\ structure is traversed node by node to the new
location, since that traversal is only needed to maintain a single \X\
structure.  Instead, the new point can be reached by greater hops, for
example by using a balanced tree with leaf nodes doubly linked as an
implementation of the \YList.  Movements larger than a single hop use
the tree to navigate to the new point.  For this structure, all
queries at or within one boundary to the current location would take
$O(k)$ time, and all others would take $O(\lg{n}+k)$.  Because of all
the \X\ structures, region insertion and deletion times would be much
greater though, $O(n\lg{n})$.  The more static nature of this structure
would allow multiple input stream locations to use one index however,
and each would only need to maintain the list for pointers to current
nodes within the structure.

\vspace*{-1ex}
\subsection{Non-spatial multi-dimensional data}
%
The focus has been on spatial data with $x$ and $y$ coordinates.
However, these techniques could be equally well applied as a general
multi-dimensional data space.  The obvious modifications could be made
and extended to $n$ dimensions as outlined above.  One important issue
to address is the order of the cascade of \List\ structures.  As
described in Section~\ref{sec:performance}, if regions are dispersed
 equally, it is generally best to move the most varying
parameter to the end of the cascade, and move the least varying to the
top.  This structure is also appropriate for range queries over a
single dimension over trending data by maintaining only the top
\YList.  Since the index sizes are relatively small, it is conceivable
that a system could consistently maintain one dimensional \id{DCT}\ 
structures, and then dynamically begins to build 2 or $n$ dimensional
structures when queries requesting such regions are instantiated.  The
advantage of this method is that the structures are no longer
maintained as the queries requesting those regions are deleted.  In
addition, statistics could be maintained by the one-dimensional
structures in order to best predict the order of cascade for
dimensions.

%% \section{Geostationary Operational Environmental Satellite Example}
%% \label{sec:goes}

%% As an example of a good input stream, this data structure was
%% discovered while developing a prototype implementation of our
%% geostreams server.  We are designing our geostreams server to be a
%% service connecting users to real-time remotely sensed data products.
%% The prototype server will allow users to specify queries using the
%% \ac{ogis} \acl{wms}.  This standard allows for only a very narrow set
%% of query operations, limited to identifying a stream and adding
%% spatial selections, temporal selections, and scale parameters. This
%% limited set satisfies a large class of remote sensing users who are
%% primarily interested in the original remotely sensed data, but cut out
%% for their specific spatial and temporal interest.  Our prototype
%% streaming data target is the \ac{NOAA} \ac{GOES} West weather satellite imager
%% sensor~\cite{noaa-goes}.  \ac{GOES} offers a continuous stream of data for
%% regions from the continental US, to a hemisphere centered near Hawaii.
%% Data from the \ac{GOES} visible channel comes in blocks that contain 8 rows
%% of data.  The number of columns in a row varies from frame to frame.
%% An entire frame of data is reported 8 rows at a time, from North to
%% South.  New frames start from their most Northern extents.  On average
%% there are about 5125 rows per frame, which means that for every frame
%% start requiring a long traverse through the \ct\ structure, there are
%% about 640 small steps downward in the $y$ direction.  Query regions
%% also tend to be approximately square regions covering relatively large
%% extents.

%% This dataset is well suited to the \ct.  In each frame, the data
%% trends only in the downward direction and incrementally.  Therefore,
%% endpoints are only added into the \X\ structure one time, limiting the
%% maintenance time to $O(n\lg{n})$.  Actually, for normal \ac{GOES} data, the
%% starting column of the each row does not change within a frame and the
%% \Y\ and \A\ structures are sufficient for determining which regions
%% overlap any given row.

%% With respect to \ac{GOES} data, the reason for extending the structure to
%% account for more general trending data has to do with projection
%% systems.  Like most remotely sensed imagery, the original data is in
%% it is own projection system, while users would like to have data
%% streamed in a projection system of their choice, \ac{utm} for example.
%% Re-projection is an expensive operation that should be avoided for
%% data not answering a specific query.  Our plan to allow users to
%% specify queries in their projection system requires have a number of
%% \ct\ structures, one for each input projection system.  All query
%% regions will be projected into the \ac{GOES} system, and roughly described
%% with a bounding box.  For each projection included with a region
%% identified in this first \ct\ structure, the bounding box of the
%% incoming rows of \ac{GOES} data will be re-projected into that system, and
%% represented with a new bounding box.  This box will be used to
%% identify the final intersections of the incoming \ac{GOES} stream with the
%% regions of interest.  In those other projection schemes, the incoming
%% rows will still trend in a smooth line, but in both $x$ and $y$
%% coordinates.

\vspace*{-1ex}
\section{Conclusions and Future Work}
\label{sec:conclusions}

In this paper, we have presented the \acf{ct}, a simple data structure
designed to follow trending geospatial data points that constitute
streaming geospatial image data.  The focus has been on a two
dimensional stabbing query, but we have offered modifications to a
number of related problems.  Theoretical and rule of thumb performance
bounds have been discussed.  Although it is difficult to quantify
performance, initial further work will focus on more experimental
tests of the \ac{ct} for various realistic scenarios.

We are currently implementing the \acf{ct} as part of query processing
architecture that supports complex continuous queries over streams of
remotely-sensed geospatial image data, so called GeoStreams. The
proposed DCT will build an important component to facilitate the
optimization of multiple queries against such a stream.
\vspace*{-1ex}
%ACKNOWLEDGMENTS are optional
\section{Acknowledgments}
This work was partially supported by the NSF grant IIS-0326517.

\vspace*{-1ex}
\bibliographystyle{abbrv}
\bibliography{geostreams}
\end{document}

Check citeseer.nj.nec.com/overmans87efficient.html

Heres the Link for the database

http://libils.ucdavis.edu/F/CP5YMEEA8E8MYLPICE9KENVFCUYPAA45YXGR35MMYEY5FCGRAU-30038?func=item-global&doc_library=UCD01&doc_number=002242745&year=&volume=&sub_library=ELECT

TODO 
Read r+-trees to verify them

Get some more expected results for the structure
  Look at experiments in zhang.

Check definition of priority tree.

http://citeseer.ist.psu.edu/mart95segmentbased.html

Citations
---------
Segment Tree

Am I right about using r_id in \Ylist?
Fix space in figures

Number steps in updates in figure
